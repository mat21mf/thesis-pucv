\documentclass[]{article}
\usepackage{lmodern}
\usepackage{amssymb,amsmath}
\usepackage{ifxetex,ifluatex}
\usepackage{fixltx2e} % provides \textsubscript
\ifnum 0\ifxetex 1\fi\ifluatex 1\fi=0 % if pdftex
  \usepackage[T1]{fontenc}
  \usepackage[utf8]{inputenc}
\else % if luatex or xelatex
  \ifxetex
    \usepackage{mathspec}
  \else
    \usepackage{fontspec}
  \fi
  \defaultfontfeatures{Ligatures=TeX,Scale=MatchLowercase}
\fi
% use upquote if available, for straight quotes in verbatim environments
\IfFileExists{upquote.sty}{\usepackage{upquote}}{}
% use microtype if available
\IfFileExists{microtype.sty}{%
\usepackage{microtype}
\UseMicrotypeSet[protrusion]{basicmath} % disable protrusion for tt fonts
}{}
\usepackage[margin=2.25cm]{geometry}
\usepackage{hyperref}
\hypersetup{unicode=true,
            pdftitle={Aplicación de técnicas de clusterización para la clasificación de grupos vegetacionales.},
            pdfauthor={Matías F. Rebolledo G.},
            pdfborder={0 0 0},
            breaklinks=true}
\urlstyle{same}  % don't use monospace font for urls
\usepackage{graphicx,grffile}
\makeatletter
\def\maxwidth{\ifdim\Gin@nat@width>\linewidth\linewidth\else\Gin@nat@width\fi}
\def\maxheight{\ifdim\Gin@nat@height>\textheight\textheight\else\Gin@nat@height\fi}
\makeatother
% Scale images if necessary, so that they will not overflow the page
% margins by default, and it is still possible to overwrite the defaults
% using explicit options in \includegraphics[width, height, ...]{}
\setkeys{Gin}{width=\maxwidth,height=\maxheight,keepaspectratio}
\IfFileExists{parskip.sty}{%
\usepackage{parskip}
}{% else
\setlength{\parindent}{0pt}
\setlength{\parskip}{6pt plus 2pt minus 1pt}
}
\setlength{\emergencystretch}{3em}  % prevent overfull lines
\providecommand{\tightlist}{%
  \setlength{\itemsep}{0pt}\setlength{\parskip}{0pt}}
\setcounter{secnumdepth}{5}
% Redefines (sub)paragraphs to behave more like sections
\ifx\paragraph\undefined\else
\let\oldparagraph\paragraph
\renewcommand{\paragraph}[1]{\oldparagraph{#1}\mbox{}}
\fi
\ifx\subparagraph\undefined\else
\let\oldsubparagraph\subparagraph
\renewcommand{\subparagraph}[1]{\oldsubparagraph{#1}\mbox{}}
\fi

%%% Use protect on footnotes to avoid problems with footnotes in titles
\let\rmarkdownfootnote\footnote%
\def\footnote{\protect\rmarkdownfootnote}

%%% Change title format to be more compact
\usepackage{titling}

% Create subtitle command for use in maketitle
\newcommand{\subtitle}[1]{
  \posttitle{
    \begin{center}\large#1\end{center}
    }
}

\setlength{\droptitle}{-2em}

  \title{Aplicación de técnicas de clusterización\\
para la clasificación de grupos vegetacionales.}
    \pretitle{\vspace{\droptitle}\centering\huge}
  \posttitle{\par}
    \author{Matías F. Rebolledo G.}
    \preauthor{\centering\large\emph}
  \postauthor{\par}
    \date{}
    \predate{}\postdate{}
  

% imagenes svg clickeables da error contra xcolor
% \usepackage{svg}


% vista landscape con rotating
\usepackage{pdflscape}
%% definicion 1
  \newcommand{\blandscape}{\begin{landscape}}
  \newcommand{\elandscape}{\end{landscape}}
%% definicion 2
% \newcommand{\blandscape}{
% \begin{landscape}
% \thispagestyle{empty}}
% \newcommand{\elandscape}{
% \vfill
% \raisebox{-2}{\makebox[\linewidth]{\thepage}}
% \end{landscape}}
% apt-get install texlive-games --no-install-recommends
% \usepackage{rotating}
% \newcommand{\brotating}{\begin{rotate}{90}}
% \newcommand{\erotating}{\end{rotate}}


% codificacion
\usepackage[utf8]{inputenc}
\usepackage[T1]{fontenc}
\usepackage[spanish]{babel}

% underscore funcione tanto en mathmode como en texmode
% https://tex.stackexchange.com/questions/62705/underscore-in-textmode-vs-mathmode
\AtBeginDocument{
  \catcode`_=12
  \begingroup\lccode`~=`_
  \lowercase{\endgroup\let~}\sb
  \mathcode`_="8000
}


\usepackage{inconsolata}
\usepackage{textcomp}
\usepackage{lmodern}
\usepackage{amsmath}
\usepackage{amsfonts}
\usepackage{amsthm}
\usepackage{pmboxdraw}

% MEJORES TABLAS
\usepackage{booktabs}

% TABLAS GRANDES
\usepackage{float}

% TABLAS LARGAS
\usepackage{longtable}

% multirow
\usepackage{multirow}

% insertar cortes a enlaces
\PassOptionsToPackage{hyphens}{url}\usepackage{hyperref}

% colores custom
\usepackage[dvipsnames,svgnames,x11names,hyperref,table]{xcolor}
% ver colores en: http://latexcolor.com/
\hypersetup{colorlinks,breaklinks,
            urlcolor=[rgb]{0,0,0.55},
            linkcolor=[rgb]{0,0,0.55}}

%%% color en tablas con cellcolor
\usepackage{colortbl}
%%% ejemplo \cellcolor{red!25}0.95

%%%%%% spacing
\usepackage{setspace}\doublespacing

%%%%%% parskip (separacion entre parrafos)
\setlength{\parskip}{0.3cm}

%%%%%%%%%%%%%%
% bibliografia
%%%%%%%%%%%%%%

%%%
%%% instalar pandoc-citeproc
%%% apt-get install pandoc-citeproc --no-install-recommends
%%%

%%%
%%% No es reconocido el codigo latex en el documento md
%%% en cuanto a bibliografia. Pendiente modificar
%%% Es colocada al final del documento por defecto
%%% eso debe cambiarse posteriormente o pasar
%%% el documento completo a latex
%%%

% bibliografia sin heading
\usepackage{etoolbox}
\patchcmd{\thebibliography}{\section*{\refname}}{}{}{}

%%%%%%%%%%%%%%

% MARGENES
% \usepackage[margin=0.60in,footskip=0.20in]{geometry}
% \usepackage[landscape,width=15in,height=11.5in,top=.5in,bottow=.5in]{geometry}

% % logo 1
% \usepackage{fancyhdr}
% \pagestyle{fancy}
% \rhead{\includegraphics[width = .05\textwidth]{figs/logoPUCV.pdf}}
% \usepackage{titling}

  % logo 2
  \pretitle{%
    \begin{center}
    \LARGE
    \includegraphics[width=6cm,height=8cm]{figs/logoPUCV.pdf}\\[\bigskipamount]
  }
  \posttitle{\end{center}}

% ajustar dimension pagina adjustpagedim
%https://tex.stackexchange.com/questions/233336/how-to-adjust-textwidth-and-textheight-to-paper-size-in-mid-document-and-heade
%\documentclass{article}
\usepackage{geometry}
\usepackage{lipsum}
\usepackage{fancyhdr}

\def\vcoef#1{#1=\dimexpr #1*\pdfpageheight/\paperheight\relax}
\def\hcoef#1{#1=\dimexpr #1*\pdfpagewidth/\paperwidth\relax}

% reajustar tamaño pagina con 2 comandos
\makeatletter
\def\adjustpagedim#1#2{%
\newpage
\pdfpagewidth=#1 \pdfpageheight=#2
\hcoef\textwidth
\vcoef\textheight
\vsize=\textheight
\@colroom=\textheight
\@colht=\textheight
\columnwidth=\textwidth
\if@twocolumn%
   \advance\columnwidth-\columnsep
   \divide\columnwidth\tw@%
   \@firstcolumntrue%
\fi%
\hsize=\columnwidth
\linewidth=\hsize
\hcoef\evensidemargin
\hcoef\oddsidemargin
\vcoef\topmargin
\vcoef\headheight
\vcoef\headsep
\vcoef\footskip
\hcoef\marginparwidth
\hcoef\marginparsep
\headwidth=\textwidth
% this is important but does nothing here
\paperwidth=\pdfpagewidth
\paperheight=\pdfpageheight}
\makeatother
% \begin{document}
%
% \thispagestyle{fancy}
% %----->I change something here
% \newlength{\classpageheight}
% \setlength{\classpageheight}{\paperheight}
% \newlength{\classpagewidth}
% \setlength{\classpagewidth}{\paperwidth}
%
% \lipsum[1-6]
%
% \adjustpagedim{17in}{18in}
% \thispagestyle{fancy}
%
% \lipsum[1-6]
%
% \adjustpagedim{\classpagewidth}{\classpageheight}
% \thispagestyle{fancy}
%
% \lipsum[1-6]
%
% \end{document}

%%%% ruta para archivos externos ignorado (pendiente)
\makeatletter
\def\input@path{{figs/}{tabs/}{tabs/latex/}{tabs/context/}{tabs/csv/}}
\makeatother

% rutas de las figuras
\graphicspath{{figs/}{figs/png/}{figs/eps/}{figs/pdf/}}

% tablas con tamaño especifico
\usepackage{array}
\newcolumntype{L}[1]{>{\raggedright\let\newline\\\arraybackslash\hspace{0pt}}m{#1}}
\newcolumntype{C}[1]{>{\centering\let\newline\\\arraybackslash\hspace{0pt}}m{#1}}
\newcolumntype{R}[1]{>{\raggedleft\let\newline\\\arraybackslash\hspace{0pt}}m{#1}}

% hyphenationes (son ignoradas)
\babelhyphenation[spanish]{asig-nación}

% hyphenationes con underscore
\newcommand{\gpus}{\_\discretionary{-}{}{}}

% Output loop---100 consecutive dead cycles
\maxdeadcycles=1000

% LaTeX Error: Too many unprocessed floats
% https://texfaq.org/FAQ-tmupfl
% usar \clearpage cada 2 imagenes

% listings
\usepackage{listings}

\begin{document}
\maketitle


% Redefinir dimensiones de documento
\newlength{\classpageheight}
\setlength{\classpageheight}{\paperheight}
\newlength{\classpagewidth}
\setlength{\classpagewidth}{\paperwidth}

% Renombrar toc sin usepackage
\renewcommand{\contentsname}{Índice general}
\renewcommand{\listtablename}{Índice de cuadros}
\renewcommand{\listfigurename}{Índice de gráficos}
\renewcommand{\refname}{Referencias}
\renewcommand{\tablename}{Cuadro}
\renewcommand{\figurename}{Gráfico}

% toc
% Llamar toc solo desde aqui o desde
% preambulo YAML
\clearpage
\tableofcontents
\listoftables
\listoffigures

\clearpage

\input{vacio}

\section{Introducción.}\label{introduccion.}

El presente informe resume las labores realizadas durante la
implementación de un método de clusterización usando el algoritmo de
k-medias para identificar grupos homogéneos de vegetación. Describe los
pasos que involucraron, preparación de la información, aplicación del
método de k-medias, estimación de métricas de desempeño para obtener el
número óptimo de clusters, evaluación de la precisión mediante el método
de matriz de confusión y generación de las imágenes resultantes
clasificadas.

Tanto en áreas de biodiversidad como en temas medio ambientales
diversos, es necesario identificar zonas que posean características
homogéneas que puedan ser separadas en sub divisiones para poder
implementar planes específicos por cada entidad en virtud de sus
características distintivas. Una de las formas es poder hacer uso de
imágenes satelitales que representen un conjunto de variables de interés
las que a su vez puedan ser agrupadas en grupos más pequeños con
características homogéneas entre sí. La imagen satelital además de ser
una fuente de datos, representa una zona determinada del espacio y es
generada mediante sensores presentes en satelites y vehículos de vuelo
especializados. Conforme la tecnología con que el sensor genera la
imagen, se tiene que ésta puede contener errores generados por las
condiciones atmosféricas como por las mismas propiedades físicas con las
que cada sensor genera la información. Por este motivo toda información
espacial debe ser cotejada con métodos en terreno para corroborar que la
información con la que se cuenta sea lo más exacta posible.

En las últimas décadas, técnicas estadísticas han sido utilizadas para
la identificación de grupos homogéneos en diferentes zonas espaciales
mediante el uso de información satelital. En nuestro estudio haremos uso
de la técnica denominada algoritmo de k-medias o \emph{k-means} en
inglés que permite separar grupos homogéneos mediante el criterio de
minimización de la suma de cuadrados dentro de cada grupo. A pesar que
esta como otras técnicas han ido adoptándose progresivamente para
responder preguntas de interés de la biodiversidad y el medio ambiente,
hemos sido testigos de un aumento en las capacidades de cómputo con un
consecuente abaratamiento en los costos de adquisición de nuevas
tecnologías. Esto ha tenido una gran implicancia en términos de
viabilidad de implementación de proyectos que requieran mecanismos
automatizados que generen información útil a partir de datos que sería
considerablemente inviables de conseguir operacionalmente, ya sea en
tiempo, costos o mano de obra.

Siguiendo esta línea, esta investigación incopora un método novedoso
para el procesamiento de información satelital que es de por si de un
tamaño elevado y que ha restringido muchos estudios en cuanto a recursos
computacionales.

\clearpage

\section{Estado del arte.}\label{estado-del-arte.}

\clearpage

\section{Marco teórico.}\label{marco-teorico.}

\subsection{Método de k-medias.}\label{metodo-de-k-medias.}

El método de k-medias fue diseñado a mediados de los años cincuenta
\footnote{El artículo fue publicado casi tres décadas después según Wikipedia.
Consultado en Septiembre 2019.} por el autor (Lloyd
\protect\hyperlink{ref-lloyd_least_1982}{1982}) y consiste en un
algoritmo iterativo que divide el conjunto de datos en particiones cuya
suma de cuadrados dentro de cada partición es mínima.

\subsubsection{Estimación puntual.}\label{estimacion-puntual.}

\subsubsection{Estimación de intervalo.}\label{estimacion-de-intervalo.}

Para cada valor de \(K (1, 2, \ldots, K^{MAX})\):

\begin{itemize}
\tightlist
\item
  Realizar \(M\) corridas (\textgreater{}30), cada una con una semilla
  al azar.
\item
  En cada una de las \(M\) corridas, calcular las medidas de desempeño
  (\(\text{R}^2\), Elbow, CH), obteniendo así \(M\) valores de cada una.
\item
  Entonces considerar la medida x (cualquiera de ellas), calcular el
  promedio \(\overline{x}\) sobre los \(M\) valores y su desviación
  estándar: \[s = \sqrt{\Sigma_{i=1}^M \frac{x_i-\overline{x}}{M-1} }\]
\item
  Luego el intervalo de confianza para el desempeño promedio de x se
  calcula como:
  \[\left[\overline{x} - z^\star \cdot \frac{s}{\sqrt{M}} , \overline{x} +
    z^\star \cdot \frac{s}{\sqrt{M}}\right]\]
\end{itemize}

Para el caso de que el intervalo tenga un 95\% de confianza, el valor de
\textbf{$z^\star$} es 1.96.

\clearpage

\section{Propuesta.}\label{propuesta.}

En esta sección describiremos la estrategia que se usará para poder
clasificar grupos vegetaciones y topográficos usando el algoritmo de
k-medias. Esta consiste en varios pasos necesarios para poder procesar
la información de manera eficiente.

\subsection{Selección de variables para identificación de grupos
homogéneos.}\label{seleccion-de-variables-para-identificacion-de-grupos-homogeneos.}

La definición de la muestra consistió en definir un área total para el
estudio y también dividir el área total en áreas más pequeñas siguiendo
un modelo de clasificación de vegetación previamente definido por los
autores (Luebert and Pliscoff
\protect\hyperlink{ref-luebert_sinopsis_2006}{2006}). Ellos definen
trece zonas vegetacionales para la Región Metropolitana y la de
Valparaíso con las cuales, definimos seis grupos que constituyen zonas
geográficas adyacentes cuya finalidad es disminuir el área total del
proyecto en zonas de menor tamaño y por otro lado, mantener homogeneidad
dada por los grupos colindantes en cuestión. Adicionalmente se crea un
grupo cero que equivale a la sumatoria de las seis zonas menores (ver
cuadro \ref{reftabdesc}).

\input{headtabdesc}

\subsubsection{Descripción de los datos del
estudio.}\label{descripcion-de-los-datos-del-estudio.}

Se seleccionaron en incialmente 6 variables para el estudio (Ver cuadro
\ref{reftabtvar}.). Por un lado, 5 de ellas se derivan de una variable
llamada modelo de elevación (DEM) y la sexta corresponde al índice de
humedad relativo en la superficie foliar (NDMI).

\input{headtabtvar}

Cada una de las 6 variables fue sometida a un proceso de normalización u
homogeneización para lograr que todas tuvieran un mismo sistema de
proyección, tamaño y resolución. Además, se incoporaron las coordenadas
latitud y longitud como las 2 primeras variables comunes a todos los
grupos para poder realizar un análisis espacial de las variables
inicialmente seleccionadas, por lo tanto, pasamos a tener un total de 8
variables por grupo en vez de 6.

Además de la homogenización de las variables fue también necesario
estandarizarlas, puesto que ellas se encuentran en unidades de medición
y escalas de magnitud diferentes.

\input{headtabsmpl}

\subsection{Desarrollo de un procedimiento automatizado de
identificación de
grupos.}\label{desarrollo-de-un-procedimiento-automatizado-de-identificacion-de-grupos.}

\newpage

\adjustpagedim{502.9mm}{388.6mm}

\subsubsection{Pre procesamiento de
imágenes.}\label{pre-procesamiento-de-imagenes.}

\begin{figure}[ht]
\begin{center}
\fbox{\includegraphics[width=.9\textwidth]{/home/matbox/Documents/TrabajosExtra/petra_biotopos/src/diag/diagrama_01_rasters.png}}

\caption{Proceso de homogenización de imágenes satelitales.}
\label{reffigrast}
\end{center}
\end{figure}


\newpage

\subsubsection{Transformación de
imágenes.}\label{transformacion-de-imagenes.}

Esta sección consiste principalmente en la conversión de imágenes
satelitales en archivos de tipo big.matrix, cuya diferencia principal es
que están dispuestos en forma de columnas, mientas que cada imagen están
en forma de matriz. El nombre big.matrix es confuso, porque no hace
relación a una matriz propiamente tal, sino a que es un data.frame, pero
uno que puede ser accedido con una tecnología de memoria compartida
(shared memory), permitiendo que en la sesión de trabajo no se consuma
memoria RAM en un monto equivalente al tamaño del archivo, sino que
continúe la memoria compartida en el disco y tener una sesión
completamente dedicada a los procesos que se ejecutarán y no a tener un
archivo tan grande levantado en la sesión. Es es uno de los pilares con
los que este trabajo pudo ser llevado a cabo.

\begin{figure}[ht]
\begin{center}
\fbox{\includegraphics[width=.7\textwidth]{/home/matbox/Documents/TrabajosExtra/petra_biotopos/src/diag/diagrama_02_xyz.png}}

\caption{Proceso de conversión de imágenes satelitales.}
\label{reffigixyz}
\end{center}
\end{figure}


\newpage

\subsubsection{Aplicación de algoritmo de
clasificación.}\label{aplicacion-de-algoritmo-de-clasificacion.}

En esta sección se utiliza el algoritmo de k-medias que realiza una
estimación iterativa en la que calcula la suma cuadrática intra clusters
al mismo tiempo que va asignando valores a cada observación como
perteneciente a un grupo cluster potencial, vuelve a calcular
corrigiendo el valor asignado a algunas observaciones y si obtiene una
menor suma cuadrática intra, lo vuelve a re calcular hasta que la
diferencia entre cada etapa no es mayor a un umbral de tolerancia, más
allá del cual se detiene puesto que converge.

\begin{figure}[ht]
\begin{center}
\fbox{\includegraphics[width=.7\textwidth]{/home/matbox/Documents/TrabajosExtra/petra_biotopos/src/diag/diagrama_03_bigmatrices.png}}

\caption{Proceso de creación de objetos de memoria compartida.}
\label{reffigbigm}
\end{center}
\end{figure}


\adjustpagedim{\classpagewidth}{\classpageheight}

\newpage

\begin{figure}[ht]
\begin{center}
\fbox{\includegraphics[width=.7\textwidth]{/home/matbox/Documents/TrabajosExtra/petra_biotopos/src/diag/diagrama_04_bigkmeans.png}}

\caption{Proceso de ejecución de k-medias con big matrices.}
\label{reffigbigk}
\end{center}
\end{figure}


\newpage

\subsubsection{Estadígrafos de evaluación del algoritmo
k-medias.}\label{estadigrafos-de-evaluacion-del-algoritmo-k-medias.}

Esta sección consiste en calcular a partir de los resultados del
procedimiento de clusterización, algunas métricas que permitan
identificar el número óptimo de clusters, entre las cuales, tenemos,
\(\text{R}^2\), Elbow, Calinski-Harabasz y Silhouette. De las cuatro, se
implementaron las tres primeras, mientras que la última tiene una
limitación dada por el tamaño muestral lo que conlleva el cálculo de
matrices de distancia de tamaño elevado.

\begin{figure}[ht]
\begin{center}
\fbox{\includegraphics[width=.8\textwidth]{/home/matbox/Documents/TrabajosExtra/petra_biotopos/src/diag/diagrama_05_estadigrafos.png}}

\caption{Proceso de extracción de estadígrafos de evaluación.}
\label{reffigesta}
\end{center}
\end{figure}


\newpage

\subsubsection{Reconversión de big matrices en imágenes
rasters.}\label{reconversion-de-big-matrices-en-imagenes-rasters.}

Esta sección consiste en la generación de nuevas imágenes satelitales a
partir de las columnas con los valores clusters obtenidos tras la
aplicación de cada algoritmo. Cada vector columna está en un forma de
tipo big.matrix y para convertirlo en imagen raster es necesario
realizar varios pasos, de los cuales el más imporante, es el de
intersección con el vector original de la imagen que contiene las
coordenadas espaciales, de manera de poder convertirlo en un objeto
Ascii grid y posteriormente en un archivo Geotiff.

\begin{figure}[ht]
\begin{center}
\fbox{\includegraphics[width=.8\textwidth]{/home/matbox/Documents/TrabajosExtra/petra_biotopos/src/diag/diagrama_06_invertir_xyz.png}}

\caption{Proceso de rasterización desde big matrices.}
\label{reffigoxyz}
\end{center}
\end{figure}


\adjustpagedim{\classpagewidth}{\classpageheight}

\subsubsection{Caracterización de
agrupamientos.}\label{caracterizacion-de-agrupamientos.}

Esta etapa consiste en la caracterización de los clusters seleccionados
para asignar nombres a los clusters elegidos. Para ello se calculan
indicadores para cada uno de los clusters, tales como, promedio,
desviación estándar, error estándar, máximo, mínimo, intervalo de
confianza, los que a su vez son graficados para analizar en detalle como
caracterizar cada agrupamiento y así ponerles un nombre distintivo a
cada uno.

Además de caracterizar cada agrupamiento, tiene otro propósito que es
permitir recodificar por tramos los valores de las variables para poder
construir una matriz de confusión y poder así evaluar qué tan bien
clasificó el algoritmo cluster cada una de las variables originales. En
el gráfico \ref{reffigmatc} está el proceso correspondiente.

\begin{figure}[ht]
\begin{center}
\input{figmatc}
\caption{Proceso de caracterización de variables según cluster.}
\label{reffigmatc}
\end{center}
\end{figure}


\clearpage

\subsubsection{Evaluación de precisión de clasificación por matriz de
confusión.}\label{evaluacion-de-precision-de-clasificacion-por-matriz-de-confusion.}

Es de interés señalar que no solo para la identificación de zonas
homogéneas con intereses de biodiversidad es posible usar técnicas
estadísticas. Según los autores (Congalton and Green
\protect\hyperlink{ref-congalton_assessing_2008}{2008}) existen
protocolos que permiten medir que tan exactas son las mediciones que se
están haciendo a través del uso de información satelital. Esto forma
parte de todo el manejo de información cartográfica espacial que sea
medida de manera indirecta y que requiera entre otras cosas visitas en
terreno para garantizar la exactitud de las áreas identificadas.

Además de tener una gran variedad de técnicas estadísticas en la
actualidad, se mencionó inicialmente que los sensores satelitales, por
su diseño, están expuestos a fuentes de perturbación lo que genera
errores en los datos medidos. Tanto esto como cualquier procedimiento
estadístico debe ser sometido a un protocolo de medición de precisión,
lo que conlleva principalmente la puesta en operación de un plan de
muestreo en terreno para la verificación de zonas de interés siguiendo
también en este caso una metodología estadística como herramienta de
planificación.

Si bien es cierto fue de interés para nuestra investigación identificar
este conocimiento como una herramienta con gran potencial para nuestros
análisis, es necesario tomar en cuenta que esto se enmarca en un
problema que es también de orden transversal en muchas áreas de la
información. Consiste principalmente en tener a disposición información
previamente tipificada. En este caso, es necesario tener a lo menos una
muestra de zonas representativas que hayan sido previamente
categorizadas a un tipo específico de zona vegetacional o de propiedades
topográficas según sea el caso. Como esto es muchas veces inviable en
términos de presupuesto y tiempo, no se cuenta con áreas previamente
clasificadas para el estudio por lo que no es posible aplicar al menos
la herramienta de medición denominada matriz de confusión, puesto que
requiere no solo de los grupos identificados mediante el algoritmo de
k-medias, sino que además requiere de las áreas nominadas por un
protocolo de muestreo en terreno o en última instancia de información
secundaria asociada a las propiedades de las zonas de interés, pero no
se cuenta con esta información.

Por su parte, el autor (G. F. (. F. Bonham-Carter and Bonham-Carter
\protect\hyperlink{ref-bonham-carter_geographic_1994}{1994}) señala que
para analizar la precisión del algoritmo de k-medias es posible hacer
uso de un índice de validez interna, tal como, \(\text{R}^2\), Elbow y
el Caliski-Harabasz. El autor (Desgraupes
\protect\hyperlink{ref-desgraupes_clustercrit_2018}{2018}) ha
implementado más de cincuenta fórmulas de validación interna y externa
para medir la precisión del algoritmo de k-medias. No obstante, existen
dos restricciones a la hora de utilizarlos con nuestros datos. Por un
lado, no es compatible con el formato \emph{bigmemory} y por el otro,
varios de ellos requieren el cálculo de la matriz de distancias, lo que
en nuestro estudio es inviable por tamaño de almacenamiento en disco
duro.

\clearpage

\section{Experimentos.}\label{experimentos.}

\subsection{Resultados de las corridas de kmedias
puntuales.}\label{resultados-de-las-corridas-de-kmedias-puntuales.}

El cuadro \ref{refkmn_grp00_var08_objetos_print} resume cada una de las
30 corridas de kmedias puntuales realizadas para el grupo del área
total, y que a su vez fue calculado para las seis áreas restantes.

\clearpage

\section{Conclusiones.}\label{conclusiones.}

Un aspecto importante a considerar antes de analizar cada conclusión
dice relación con las cosas que no pudieron ser implementadas en el
estudio. Entre ellas la más importante talvez es el índice
\emph{Silhouette} que es muy conocido y que permite indentificar con
mayor exactitud el número óptimo de clusters. La restricción principal
está en la matriz de distancia y en toda librería donde existe una
versión de este coeficiente es esta matriz la que se usa como insumo
para su cálculo. Otra versión podría diseñarse que no requiera crear en
disco duro dicha matriz.

Otro aspecto importante para casos futuros está el uso del algoritmo
Batch-k-means e incluso Mini-batch-k-means.

\clearpage

\blandscape

\section*{Anexos.}\label{anexos.}
\addcontentsline{toc}{section}{Anexos.}

\subsection*{Resumen cuadros indicadores k-medias
puntuales.}\label{resumen-cuadros-indicadores-k-medias-puntuales.}
\addcontentsline{toc}{subsection}{Resumen cuadros indicadores k-medias
puntuales.}

\input{headkmn_grp00_var08_objetos_print}

\elandscape

\blandscape

\input{headkmn_grp01_var08_objetos_print}

\elandscape

\blandscape

\input{headkmn_grp02_var08_objetos_print}

\elandscape

\blandscape

\input{headkmn_grp03_var08_objetos_print}

\elandscape

\blandscape

\input{headkmn_grp04_var08_objetos_print}

\elandscape

\blandscape

\input{headkmn_grp05_var08_objetos_print}

\elandscape

\blandscape

\input{headkmn_grp06_var08_objetos_print}

\elandscape

\subsection*{Resumen gráficos indicadores k-medias
puntuales.}\label{resumen-graficos-indicadores-k-medias-puntuales.}
\addcontentsline{toc}{subsection}{Resumen gráficos indicadores k-medias
puntuales.}

\section*{Referencias.}\label{referencias.}
\addcontentsline{toc}{section}{Referencias.}

\subsection*{Enlaces consultados.}\label{enlaces-consultados.}
\addcontentsline{toc}{subsection}{Enlaces consultados.}

\subsubsection*{Medición de precisión en mapas
temáticos.}\label{medicion-de-precision-en-mapas-tematicos.}
\addcontentsline{toc}{subsubsection}{Medición de precisión en mapas
temáticos.}

\begin{itemize}
\tightlist
\item
  \href{https://www.researchgate.net/post/What_is_the_best_approach_to_evaluate_an_unsupervised_classification}{What
  is the best approach to evaluate an unsupervised classification}
\item
  \href{https://stats.stackexchange.com/questions/21807/evaluation-measures-of-goodness-or-validity-of-clustering-without-having-truth}{Evaluation
  measures of goodness or validity of clustering without having truth}
\item
  \href{https://www.researchgate.net/post/How_to_measure_classification_accuracy_in_the_absence_of_ground_truth}{How
  to measure classification accuracy in the absence of ground truth}
\end{itemize}

\subsubsection*{Creación de matrices de distancia de gran
tamaño.}\label{creacion-de-matrices-de-distancia-de-gran-tamano.}
\addcontentsline{toc}{subsubsection}{Creación de matrices de distancia
de gran tamaño.}

\begin{itemize}
\tightlist
\item
  \href{https://stackoverflow.com/questions/44313983/r-distm-with-big-memory/44321935\#44321935}{R:
  distm with Big Memory}
\end{itemize}

\subsubsection*{Algoritmos usados en la
tesis.}\label{algoritmos-usados-en-la-tesis.}
\addcontentsline{toc}{subsubsection}{Algoritmos usados en la tesis.}

\begin{itemize}
\tightlist
\item
  \href{https://mlpack.org/doc/mlpack-git/doxygen/kmtutorial.html}{K-Means
  tutorial - MLpack}
\end{itemize}

\clearpage

\subsection*{Artículos consultados.}\label{articulos-consultados.}
\addcontentsline{toc}{subsection}{Artículos consultados.}

\providecommand\BIBentryALTinterwordstretchfactor{2.5}

\bibliographystyle{apalike}\bibliography{resumen}

\hypertarget{refs}{}
\hypertarget{ref-bonham-carter_geographic_1994}{}
Bonham-Carter, G. F. (Graeme Francis), and Graeme Bonham-Carter. 1994.
\emph{Geographic Information Systems for Geoscientists: Modelling with
GIS}. Elsevier.

\hypertarget{ref-congalton_assessing_2008}{}
Congalton, Russell G., and Kass Green. 2008. \emph{Assessing the
Accuracy of Remotely Sensed Data: Principles and Practices, Second
Edition (Mapping Science)}. 2nd ed. Mapping Science. CRC Press.
\url{http://gen.lib.rus.ec/book/index.php?md5=CE1869245D6EBBBC1F67CB64E949347E}.

\hypertarget{ref-desgraupes_clustercrit_2018}{}
Desgraupes, Bernard. 2018. \emph{clusterCrit: Clustering Indices}.
\url{https://CRAN.R-project.org/package=clusterCrit}.

\hypertarget{ref-lloyd_least_1982}{}
Lloyd, S. 1982. ``Least Squares Quantization in PCM.'' \emph{IEEE
Transactions on Information Theory} 28 (2): 129--37.
doi:\href{https://doi.org/10.1109/TIT.1982.1056489}{10.1109/TIT.1982.1056489}.

\hypertarget{ref-luebert_sinopsis_2006}{}
Luebert, Federico, and Patricio Pliscoff. 2006. \emph{Sinopsis
Bioclimática Y Vegetacional de Chile}. Editorial Universitaria.


\end{document}
